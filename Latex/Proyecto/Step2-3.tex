En esta sección, primero se procedió a verificar la consistencia de las volatilidades entregadas ($\sigma$), las cuales serán utilizadas posteriormente en el cálculo del precio de las opciones (\textit{Option Fair Value}). Para el caso de las volatilidades, se procedió a descomponer los diferentes $\sigma$, en términos de \textit{Risk Reversal} y \textit{Butterfly} utilizando las siguientes ecuaciones:

\begin{equation}
    \sigma_{Call}= \sigma_{ATM}+\sigma_{BF}+\frac{\sigma_{RR}}{2}
\end{equation}
\begin{equation}
    \sigma_{Put}= \sigma_{ATM}+\sigma_{BF}-\frac{\sigma_{RR}}{2}
\end{equation}

\noindent Las cuales, en términos de sus $\Delta$, son consistentes con las volatilidades entregadas en los datos del documento \textit{Excel}, presentando un error prácticamente de cero, tal como se puede observar en las tablas anexadas 5 y 6. \\\\
\noindent Una vez obtenidas las volatilidades, se procedió a utilizarlas para calcular el precio de las opciones, tanto para los diferentes \textit{Tenores}, como para los diversos $\Delta$, mediante la formula de Black-Scholes, la cual de forma general se puede expresar como:

\begin{equation}
    V_0= \epsilon \cdot S_0 \cdot e^{-qT} \cdot N(\epsilon \cdot d_1) - \epsilon \cdot K \cdot e^{-rT} \cdot N(\epsilon \cdot d_2)
    \label{BlackScholes}
\end{equation}
\noindent En donde:
\begin{equation*}
    d_1= \frac{\ln(\frac{S_0}{K})+(r-q)\cdot T}{\sigma \cdot \sqrt{T}}+ \frac{\sigma \cdot \sqrt{T}}{2}
\end{equation*}
\begin{equation*}
    d_2=d_1-\sigma \cdot \sqrt{T}
\end{equation*}

     \begin{equation*}
     \label{eq:aqui-le-mostramos-como-hacerle-la-llave-grande}
     \epsilon = \left\{
	       \begin{array}{ll}
		 +1      &  si\ es\ una\ opción\ Call\\
		 -1 &  si\;es\;una\;opción\;Put\\
		 
	       \end{array}
	     \right.
   \end{equation*}
   
   \begin{equation*}
       N = Funci\acute{o}n\;de\;Densidad\;Acumulada
   \end{equation*}
\noindent Finalmente, realizando el procedimiento anterior se obtuvo un error promedio de 0.0934, lo que, dado los relativamente diminutos valores de las opciones puede ser considerado un valor a tomar en cuenta, según la comparación entre los valores calculados y los valores teóricos presentados en el documento \textit{Excel}. Estos errores se muestran en las tablas anexadas 7 y 8 respectivamente, con una muestra de 50 valores en conjunto a su error respectivo.
\newpage

