\noindent Existen muchas formas de calibrar un modelo, cada uno contando con diversas estrategias y herramientas de calibración, así como un comportamiento y reacción específicas a las volatilidades y precios de mercado. \\\\
\noindent En este trabajo, se presentó una de muchas formas existentes para valorizar derivados financieros, sin embargo, tal como se muestra en los resultados expuestos en este informe, no se logra llegar a los mismos resultados que los de mercado, esto debido a que el mercado no se guía por reglas específicas al momento de la valorización de estos instrumentos, aunque sin embargo, siguen comportamientos racionales que pueden ser parcialmente plasmados en ecuaciones tales como las estudiadas en este curso, como lo pueden ser el modelo de \textit{Black-Scholes}, o como el modelo de volatilidad estocástica estudiado en este informe, el modelo de \textit{Heston}. Como conclusión, podemos decir que la importancia de entender estos diversos fenómenos y modelos financieros no radica en obtener un modelo que logre valorizar opciones de igual manera, sino entender como podemos plasmar las pautas que siguen los mercados para valorizar estos activos en las matemáticas, entender como se guían los distintos tipos de movimientos que toman los instrumentos financieros, y saber que herramientas son más útiles para describir como se comportan estos, ya sea según el requerimiento que sera necesario computacionalmente para realizar estos cálculos, o así como el costo en el que se deberá incurrir para obtener estos, siendo uno de los principales puntos abordados en este informe.
\newpage





